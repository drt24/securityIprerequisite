\documentclass[12pt,a4paper]{article}
\usepackage[utf8]{inputenc}
\usepackage{amsmath}
\usepackage{amsfonts}
\usepackage{amssymb}
\usepackage{graphicx}
\usepackage[left=2cm,right=2cm,top=2cm,bottom=2cm]{geometry}

% Define the exercise command for efficiently creating an question with a solution.
\newcounter{questioncounter}
\newcounter{answercounter}
\newenvironment{question}{
\refstepcounter{questioncounter}\textbf{Question \thequestioncounter}
\newline}{\vspace{1em}\newline}
\newenvironment{answer}{
\refstepcounter{answercounter}\textbf{Answer \theanswercounter}
\newline}{\vspace{1em}\newline}

\newif\ifshowanswers
\showanswersfalse

\newcommand{\exercise}[2]{%
\ifshowanswers%
\begin{answer}
#2
\end{answer}
\else%
\begin{question}
#1
\end{question}
\fi}

\author{Daniel Thomas}
\title{Security I -- Mathematical prerequisites}

\begin{document}
\maketitle

\begin{abstract}
The Security I course relies on some mathematical background which students may not be familiar with.
This document seeks to cover that material and provide exercises which ensure that the material is well understood so that students can be confident that they will understand Security I.
I will provide some pointers to material which explains the concepts and a collection of exercises with solutions which should solidify understanding of the concepts.
\end{abstract}

\section{Exercises}
\showanswersfalse
\subsection*{Bookwork}
$n$ is an integer, $p$ is prime.\\
\exercise{If $A_i (1 \le i \le n)$ are finite what is $|A_1 \times A_2 \times \cdots \times A_n|$}{$|A_1|\cdot|A_2|\cdot\cdots\cdot|A_n|$}
\exercise{$A$ is finite, what is $|A^n|$ in terms of $|A|$?}{$|A|^n$}
\exercise{What is $A^*$?}{$\bigcup^\infty_{i=0}A^i$}
\exercise{$|\mathrm{Perm}(A)| =$?}{$|A|!$}
\exercise{How many functions are there of the form $f : A \rightarrow B$}{$|B^A| = |B|^{|A|}$}
\exercise{What is a group?}{\slidesgroups}
\exercise{What is an abelian group?}{\slidesgroups}
\exercise{What kind of group is $(G,\bullet)$ if it has no inverse element?}{monoid}
\exercise{What kind of group is $(\{0,1\}^n,\oplus)$?}{abelian}
\exercise{Give two abelian and two monoid groups not previously mentioned.}{\slidesgroups}
\exercise{What are the requirements for a ring?}{\slidesrings}
\exercise{What are the requirements for a field?}{\slidesfields}
\exercise{Give an example of a ring and of a field.}{\slidesrings}
\exercise{What does $n|(a-b)$ mean?}{$n$ divides $(a-b)$}
\exercise{Under what condition does $a^{p-1}\equiv 1 \nmod{p}$?}{$\gcd(a,p)=1$}
\exercise{$a \in \Z_n$, under what condition does $a^{-1}$ exist?}{$\gcd(a,n)=1$}
\exercise{What is $\Z_n^*$?}{The set of all elements of $\Z_n$ that have an inverse.}
\exercise{How do you generate $\Z_p^*$ with $g$?}{$\Z_p^* = \{g^i\!\!\mod p|0\le i\le p -2\}$}

\subsection*{Usage}
\exercise{question2}{answer2}
\subsection*{Understanding}
\exercise{question3}{answer3}


\section{Solutions}
\showanswerstrue
\subsection*{Bookwork}
$n$ is an integer, $p$ is prime.\\
\exercise{If $A_i (1 \le i \le n)$ are finite what is $|A_1 \times A_2 \times \cdots \times A_n|$}{$|A_1|\cdot|A_2|\cdot\cdots\cdot|A_n|$}
\exercise{$A$ is finite, what is $|A^n|$ in terms of $|A|$?}{$|A|^n$}
\exercise{What is $A^*$?}{$\bigcup^\infty_{i=0}A^i$}
\exercise{$|\mathrm{Perm}(A)| =$?}{$|A|!$}
\exercise{How many functions are there of the form $f : A \rightarrow B$}{$|B^A| = |B|^{|A|}$}
\exercise{What is a group?}{\slidesgroups}
\exercise{What is an abelian group?}{\slidesgroups}
\exercise{What kind of group is $(G,\bullet)$ if it has no inverse element?}{monoid}
\exercise{What kind of group is $(\{0,1\}^n,\oplus)$?}{abelian}
\exercise{Give two abelian and two monoid groups not previously mentioned.}{\slidesgroups}
\exercise{What are the requirements for a ring?}{\slidesrings}
\exercise{What are the requirements for a field?}{\slidesfields}
\exercise{Give an example of a ring and of a field.}{\slidesrings}
\exercise{What does $n|(a-b)$ mean?}{$n$ divides $(a-b)$}
\exercise{Under what condition does $a^{p-1}\equiv 1 \nmod{p}$?}{$\gcd(a,p)=1$}
\exercise{$a \in \Z_n$, under what condition does $a^{-1}$ exist?}{$\gcd(a,n)=1$}
\exercise{What is $\Z_n^*$?}{The set of all elements of $\Z_n$ that have an inverse.}
\exercise{How do you generate $\Z_p^*$ with $g$?}{$\Z_p^* = \{g^i\!\!\mod p|0\le i\le p -2\}$}

\subsection*{Usage}
\exercise{question2}{answer2}
\subsection*{Understanding}
\exercise{question3}{answer3}

\end{document}