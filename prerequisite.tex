\documentclass[12pt,a4paper]{article}
\usepackage[utf8]{inputenc}
\usepackage{amsmath}
\usepackage{amsfonts}
\usepackage{amssymb}
\usepackage{graphicx}
\usepackage[left=2cm,right=2cm,top=2cm,bottom=2cm]{geometry}

\RequirePackage[date=terse,isbn=true,doi=true,url=true,maxbibnames=9,backref=true]{biblatex}
\addbibresource{prerequisite}
\renewcommand{\bibfont}{\small}

\AtEveryBibitem{% Clean up the bibtex rather than editing it
 \clearname{editor} % remove editors
}

% Define the exercise command for efficiently creating an question with a solution.
\newcounter{questioncounter}
\newcounter{answercounter}
\newenvironment{question}{
\refstepcounter{questioncounter}\textbf{Question \thequestioncounter}
\newline}{\vspace{1em}\newline}
\newenvironment{answer}{
\refstepcounter{answercounter}\textbf{Answer \theanswercounter}
\newline}{\vspace{1em}\newline}

\newif\ifshowanswers
\showanswersfalse

\newcommand{\exercise}[2]{%
\ifshowanswers%
\begin{answer}
#2
\end{answer}
\else%
\begin{question}
#1
\end{question}
\fi}

% Security 1 slide number references
\newcommand{\slidesinitialmath}{14--20}
\newcommand{\slidesprobability}{25--26}
\newcommand{\slidesbirthday}{66}
\newcommand{\slidesall}{\slidesinitialmath, \slidesprobability\ and \slidesbirthday}

\author{Daniel Thomas}
\title{Security I -- Mathematical prerequisites}

\begin{document}
\maketitle

\begin{abstract}
The Security I course relies on some mathematical background which students may not be familiar with.
This document seeks to cover that material and provide exercises which ensure that the material is well understood so that students can be confident that they will understand Security I.
I will provide some pointers to material which explains the concepts and a collection of exercises with solutions which should solidify understanding of the concepts.
\end{abstract}

The mathematics which you need to be comfortable with for Security I is discrete mathematics and probability.
Some of this is dull -- you need to learn some notation -- while other parts are more interesting.
The exercises will fall into three categories, bookwork questions (e.g.\ on definitions), usage questions where you will apply understanding to example problems and deeper understanding questions where proofs etc.\ will be required.
Answers will be provided (or cited) for all questions.
Clearly it is to your benefit to do the questions before looking at the answers.

{\bf Aim:} Students should be comfortable with working with these concepts in general and applying them in practise, both for use in cryptography and in other parts of Computer Science.

\section*{Material}
The material to be covered is: notation, groups, rings, fields $\mathrm{GF}(2^n)$, functions, modular arithmetic, XOR, probability, birthday problem, permutations and random mappings.

Other people (including the course lecturer) have written texts which explain this material better than I can.
The Security I slides~\cite{SecurityISlides} cover the material on slides \slidesall.
If you want a longer exposition Katz and Lindell's `Introduction to Modern Cryptography'~\cite{Katz2008} is highly recommended by the lecturer and the material is covered in Appendices A and B.
Denning's `Cryptography and data security'~\cite{Denning1982} has a longer exposition and more exercises in Section 1.6.
Knuth's `The Art of Computer Programming' covers some of the material in `Volume 1 -- Fundamental Algorithms'~\cite{KnuthTAOCP1} in Sections 1.2.4 and 1.2.5; and in `Volume 2 -- Seminumerical Algorithms'~\cite{KnuthTAOCP2} in Sections 4.6.\{0,1,2\} again with exercises.

\section*{Exercises}
\showanswersfalse
\subsection*{Bookwork}
$n$ is an integer, $p$ is prime.\\
\exercise{If $A_i (1 \le i \le n)$ are finite what is $|A_1 \times A_2 \times \cdots \times A_n|$}{$|A_1|\cdot|A_2|\cdot\cdots\cdot|A_n|$}
\exercise{$A$ is finite, what is $|A^n|$ in terms of $|A|$?}{$|A|^n$}
\exercise{What is $A^*$?}{$\bigcup^\infty_{i=0}A^i$}
\exercise{$|\mathrm{Perm}(A)| =$?}{$|A|!$}
\exercise{How many functions are there of the form $f : A \rightarrow B$}{$|B^A| = |B|^{|A|}$}
\exercise{What is a group?}{\slidesgroups}
\exercise{What is an abelian group?}{\slidesgroups}
\exercise{What kind of group is $(G,\bullet)$ if it has no inverse element?}{monoid}
\exercise{What kind of group is $(\{0,1\}^n,\oplus)$?}{abelian}
\exercise{Give two abelian and two monoid groups not previously mentioned.}{\slidesgroups}
\exercise{What are the requirements for a ring?}{\slidesrings}
\exercise{What are the requirements for a field?}{\slidesfields}
\exercise{Give an example of a ring and of a field.}{\slidesrings}
\exercise{What does $n|(a-b)$ mean?}{$n$ divides $(a-b)$}
\exercise{Under what condition does $a^{p-1}\equiv 1 \nmod{p}$?}{$\gcd(a,p)=1$}
\exercise{$a \in \Z_n$, under what condition does $a^{-1}$ exist?}{$\gcd(a,n)=1$}
\exercise{What is $\Z_n^*$?}{The set of all elements of $\Z_n$ that have an inverse.}
\exercise{How do you generate $\Z_p^*$ with $g$?}{$\Z_p^* = \{g^i\!\!\mod p|0\le i\le p -2\}$}

\subsection*{Usage}
\exercise{question2}{answer2}
\subsection*{Understanding}
\exercise{question3}{answer3}

\pagebreak

\section*{Solutions}
\showanswerstrue
\subsection*{Bookwork}
$n$ is an integer, $p$ is prime.\\
\exercise{If $A_i (1 \le i \le n)$ are finite what is $|A_1 \times A_2 \times \cdots \times A_n|$}{$|A_1|\cdot|A_2|\cdot\cdots\cdot|A_n|$}
\exercise{$A$ is finite, what is $|A^n|$ in terms of $|A|$?}{$|A|^n$}
\exercise{What is $A^*$?}{$\bigcup^\infty_{i=0}A^i$}
\exercise{$|\mathrm{Perm}(A)| =$?}{$|A|!$}
\exercise{How many functions are there of the form $f : A \rightarrow B$}{$|B^A| = |B|^{|A|}$}
\exercise{What is a group?}{\slidesgroups}
\exercise{What is an abelian group?}{\slidesgroups}
\exercise{What kind of group is $(G,\bullet)$ if it has no inverse element?}{monoid}
\exercise{What kind of group is $(\{0,1\}^n,\oplus)$?}{abelian}
\exercise{Give two abelian and two monoid groups not previously mentioned.}{\slidesgroups}
\exercise{What are the requirements for a ring?}{\slidesrings}
\exercise{What are the requirements for a field?}{\slidesfields}
\exercise{Give an example of a ring and of a field.}{\slidesrings}
\exercise{What does $n|(a-b)$ mean?}{$n$ divides $(a-b)$}
\exercise{Under what condition does $a^{p-1}\equiv 1 \nmod{p}$?}{$\gcd(a,p)=1$}
\exercise{$a \in \Z_n$, under what condition does $a^{-1}$ exist?}{$\gcd(a,n)=1$}
\exercise{What is $\Z_n^*$?}{The set of all elements of $\Z_n$ that have an inverse.}
\exercise{How do you generate $\Z_p^*$ with $g$?}{$\Z_p^* = \{g^i\!\!\mod p|0\le i\le p -2\}$}

\subsection*{Usage}
\exercise{question2}{answer2}
\subsection*{Understanding}
\exercise{question3}{answer3}

\printbibliography

\end{document}