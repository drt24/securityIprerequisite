\subsection*{Bookwork}
$n$ is an integer, $p$ is prime.\\
\exercise{If $A_i (1 \le i \le n)$ are finite what is $|A_1 \times A_2 \times \cdots \times A_n|$}{$|A_1|\cdot|A_2|\cdot\cdots\cdot|A_n|$}
\exercise{$A$ is finite, what is $|A^n|$ in terms of $|A|$?}{$|A|^n$}
\exercise{What is $A^*$?}{$\bigcup^\infty_{i=0}A^i$}
\exercise{$|\mathrm{Perm}(A)| =$?}{$|A|!$}
\exercise{How many functions are there of the form $f : A \rightarrow B$}{$|B^A| = |B|^{|A|}$}
\exercise{What is a group?}{\slidesgroups}
\exercise{What is an abelian group?}{\slidesgroups}
\exercise{What kind of group is $(G,\bullet)$ if it has no inverse element?}{monoid}
\exercise{What kind of group is $(\{0,1\}^n,\oplus)$?}{abelian}
\exercise{Give two abelian and two monoid groups not previously mentioned.}{\slidesgroups}
\exercise{What are the requirements for a ring?}{\slidesrings}
\exercise{What are the requirements for a field?}{\slidesfields}
\exercise{Give an example of a ring and of a field.}{\slidesrings}
\exercise{What does $n|(a-b)$ mean?}{$n$ divides $(a-b)$}
\exercise{Under what condition does $a^{p-1}\equiv 1 \nmod{p}$?}{$\gcd(a,p)=1$}
\exercise{$a \in \Z_n$, under what condition does $a^{-1}$ exist?}{$\gcd(a,n)=1$}
\exercise{What is $\Z_n^*$?}{The set of all elements of $\Z_n$ that have an inverse.}
\exercise{How do you generate $\Z_p^*$ with $g$?}{$\Z_p^* = \{g^i\!\!\mod p|0\le i\le p -2\}$}
\exercise{What does it mean for a polynomial to be irreducible?}{\slidesgalois}
\exercise{What is $\mathrm{GF}(p^n)$?}{\slidesgalois}
\exercise{How is $\mathrm{GF}(2^n)$ implemented?}{\slidesgf}
\exercise{Are operations on $\mathrm{GF}(2^n)$ efficient to compute?}{Yes on hardware which supports it. \slidesgf}
\exercise{What is the definition of perfect secrecy?}{\slidesperfect}
\exercise{State Bayes theorem}{\slidesperfect}
\exercise{What makes a cipher unconditionally secure?}{\slidesunconditional}
\exercise{What is the birthday problem?}{\slidesbirthdayproblem}

\subsection*{Usage}
\exercise{Why is exponentiation $a^b \nmod{n}$ faster to compute than $a^b$ when $a^b > n$?}{You can reduce mod $n$ as you go along~\cite[p39, Figure 1.20]{Denning1982}}
\exercise{$3^5 \mod 7 =$ ?}{5}
\exercise{$3^{-1} \mod 10 =$ ?}{7}
\exercise{What are all the elements in $\Z^*_{19}$ of multiplicative order 18 (i.e.\ when used as a generator they generate 18 elements)?\cite[Ex 2.26]{CryptoI}}{2, 3}
\exercise{As above but for $\Z^*_{17}$ of multiplicative order 16}{TODO}

\subsection*{Understanding}
\exercise{Birthday problem: prove that as $n \rightarrow \infty$ the expected number of balls needed for a collision is $\sqrtc{n\pi/2}$}{\cite[\S A.4, p496]{Katz2008}}
\exercise{Develop an algorithm to compute $a^b \nmod{N}$\cite[p515, B.3]{Katz2008}}{\cite[p507, Algorithm B.13]{Katz2008}}
\exercise{Show how to determine that an $n$-bit string is in $\Z_N^*$\cite[p515, B.4]{Katz2008}}{TODO}