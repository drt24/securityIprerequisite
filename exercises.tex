\subsection*{Bookwork}
$n$ is an integer, $p$ is prime.\\
\exercise{If $A_i (1 \le i \le n)$ are finite what is $|A_1 \times A_2 \times \cdots \times A_n|$}{$|A_1|\cdot|A_2|\cdot\cdots\cdot|A_n|$}
\exercise{$A$ is finite, what is $|A^n|$ in terms of $|A|$?}{$|A|^n$}
\exercise{What is $A^*$?}{$\bigcup^\infty_{i=0}A^i$}
\exercise{$|\mathrm{Perm}(A)| =$?}{$|A|!$}
\exercise{How many functions are there of the form $f : A \rightarrow B$}{$|B^A| = |B|^{|A|}$}
\exercise{What is a group?}{\slidesgroups}
\exercise{What is an abelian group?}{\slidesgroups}
\exercise{What kind of group is $(G,\bullet)$ if it has no inverse element?}{monoid}
\exercise{What kind of group is $(\{0,1\}^n,\oplus)$?}{abelian}
\exercise{Give two abelian and two monoid groups not previously mentioned.}{\slidesgroups}
\exercise{What are the requirements for a ring?}{\slidesrings}
\exercise{What are the requirements for a field?}{\slidesfields}
\exercise{Give an example of a ring and of a field.}{\slidesrings}
\exercise{What does $n|(a-b)$ mean?}{$n$ divides $(a-b)$}
\exercise{Under what condition does $a^{p-1}\equiv 1 \nmod{p}$?}{$\gcd(a,p)=1$}
\exercise{$a \in \Z_n$, under what condition does $a^{-1}$ exist?}{$\gcd(a,n)=1$}
\exercise{What is $\Z_n^*$?}{The set of all elements of $\Z_n$ that have an inverse.}
\exercise{How do you generate $\Z_p^*$ with $g$?}{$\Z_p^* = \{g^i\!\!\mod p|0\le i\le p -2\}$}
\exercise{What does it mean for a polynomial to be irreducible?}{\slidesgalois}
\exercise{What is $\mathrm{GF}(p^n)$?}{\slidesgalois}
\exercise{How is $\mathrm{GF}(2^n)$ implemented?}{\slidesgf}
\exercise{Are operations on $\mathrm{GF}(2^n)$ efficient to compute?}{Yes on hardware which supports it. \slidesgf}
\exercise{What is the definition of perfect secrecy?}{\slidesperfect}
\exercise{State Bayes theorem}{\slidesperfect}
\exercise{What makes a cipher unconditionally secure?}{\slidesunconditional}
\exercise{What is the birthday problem?}{\slidesbirthdayproblem}

\subsection*{Usage}
\exercise{Show that $(\Z,+)$ is an abelian group}{closure and associativity by definition, 0 is neutral, $-x$ is inverse.}
\exercise{Show that $(\Z,\cdot)$ is monoid}{closure and associativity by definition, 1 is neutral but there is no way of making an element smaller as all elements are 1 or greater (or perhaps 0 which only maps us to 0)}
\exercise{Why is exponentiation $a^b \nmod{n}$ faster to compute than $a^b$ when $a^b > n$?}{You can reduce mod $n$ as you go along~\cite[p39, Figure 1.20]{Denning1982}}
\exercise{$53 \times 62 \mod 11 =$?}{8}
\exercise{$48 \times 73 \mod 7 =$?}{4}
\exercise{$33 \times 19 \mod 13 =$?}{3}
\exercise{$3^5 \mod 7 =$ ?}{5}
\exercise{$4^4 \mod 7 =$ ?}{4}
\exercise{$2^6 \mod 11 =$ ?}{9}
\exercise{$3^{-1} \mod 10 =$ ?}{7}
\exercise{$5^{-1} \mod 13 =$ ?}{8}
\exercise{$7^{-1} \mod 19 =$ ?}{11}
\exercise{What are all the elements in $\Z^*_{19}$ of multiplicative order 18 (i.e.\ when used as a generator they generate 18 elements)?\cite[Ex 2.26]{CryptoI}}{2, 3, 10, 13, 14, 15}
\exercise{As above but for $\Z^*_{13}$ of multiplicative order 12}{2, 6, 7, 11}
\exercise{As above but for $\Z^*_{11}$ of multiplicative order 10}{2, 6, 7, 8}
\exercise{In the Galois field GF($2^8$) modulo $x^8 + x^4 + x + 1$ calculate:
\begin{enumerate}
 \item the difference 1100 1010 minus 1001 0011;
 \item the product 0100 1011 times 0000 1001.
\end{enumerate}[2013 P4 Q8 a]}{\begin{enumerate}
 \item in GF($2^n$) subtraction is the same as bitwise XOR: 0101 1001.
 \item Binary multiplication without carry, then subtract (XOR) multiples (shifted versions) of $x^8 + x^4 + x + 1$ to elimiate any leading bits that make the result larger than 1111 1111: 0011 0101
\end{enumerate}[from the solution notes]}
\exercise{Let $a = 100 (x^2)$ in GF($2^3$) with modulus $p(x) = 1011 (x^3 + x + 1)$. Divide 1 0000 0000 0000 by 1011 to show that $a^{-1} = 100^6 \mod 1011 = 111$.~\cite[Ex~1.19, p56]{Denning1982}}{Show that question.}
\exercise{Let $a = 101$. If $a$ is squred in GF($2^3$) with irreducible polynomial $p(x) = x^3 + x + 1$ (1011 in binary), the product $d = a \times a =$ ?~\cite[p51]{Denning1982}}{111}
\exercise{Let $a = 111$ and $b = 100$. What is the product $d = a \times b$ computed in GF($2^3$) with irreducible polynomial $p(x) = 1011$ $(x^3 + x + 1)$?~\cite[p51]{Denning1982}}{1. So $a$ and $b$ are inverses mod 1011.}
\exercise{A bag contains one ball which, equiprobably, may be black or white.
A white ball is added to the bag which is then shaken.
One ball is retrieved at random and found to be white.
What is the probability that the other ball is white?~\cite[Ex~II.1, p2.12]{King2007}}{TODO}
\exercise{One person in 1000 is know not suffer from Nerd's Syndrome (a pathological inability to resist playing computer games).
A standard test is such that 99\% of those who suffer from Nerd's Syndrome show a positive result.
The same test also (falsely) shows positive with 2\% of \emph{non}-suffers.

A person selected at random is tested and found positive.
What is the probability that this person actually suffers from Nerd's Syndrome?~\cite[Ex~II.3, p2.13]{King2007}}{TODO}
\exercise{Four people go to an oyster card swapping party (they want to be difficult to track).
The oyster cards are thrown into a hat, shuffled, and then distributed randomly one to each participant.
What is the probability that a participant goes home with his original card?~\cite[Ex~II.8, p2.14]{King2007}}{TODO}
\exercise{Birthday problem: with 1024 bins how many balls are needed for the probability of collision to exceed 0.5? (You might want a calculator/computer for this one)}{38}

\subsection*{Understanding}
\exercise{Birthday problem: prove that as $n \rightarrow \infty$ the expected number of balls needed for a collision is $\sqrtc{n\pi/2}$}{\cite[\S A.4, p496]{Katz2008}}
\exercise{Develop an algorithm to compute $a^b \nmod{N}$\cite[p515, B.3]{Katz2008}}{\cite[p507, Algorithm B.13]{Katz2008}}
\exercise{Show how to determine that an $n$-bit string is in $\Z_N^*$\cite[p515, B.4]{Katz2008}}{TODO}
\exercise{Prove Bayes' Therom.}{\cite[Thrm~A.8, p495]{Katz2008}}